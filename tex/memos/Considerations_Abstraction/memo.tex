\documentclass[a4paper,UKenglish,cleveref, autoref]{lipics-v2019}
\usepackage{proof}
\usepackage{gensymb}
%This is a template for producing LIPIcs articles. 
%See lipics-manual.pdf for further information.
%for A4 paper format use option "a4paper", for US-letter use option "letterpaper"
%for british hyphenation rules use option "UKenglish", for american hyphenation rules use option "USenglish"
%for section-numbered lemmas etc., use "numberwithinsect"
%for enabling cleveref support, use "cleveref"
%for enabling cleveref support, use "autoref"


%\graphicspath{{./graphics/}}%helpful if your graphic files are in another directory

\bibliographystyle{plainurl}% the mandatory bibstyle

\title{Towards a Formalisation of Cook's Theorem: Developing Abstractions} %TODO Please add

\titlerunning{}%optional, please use if title is longer than one line

\author{Lennard Gäher}{Saarland University, Germany}{s8legaeh@stud.uni-saarland.de}{}{}%TODO mandatory, please use full name; only 1 author per \author macro; first two parameters are mandatory, other parameters can be empty. Please provide at least the name of the affiliation and the country. The full address is optional

\authorrunning{L. Gäher}%TODO mandatory. First: Use abbreviated first/middle names. Second (only in severe cases): Use first author plus 'et al.'

\Copyright{Lennard Gäher}%TODO mandatory, please use full first names. LIPIcs license is "CC-BY";  http://creativecommons.org/licenses/by/3.0/

%\ccsdesc[100]{General and reference~General literature}
%\ccsdesc[100]{General and reference}%TODO mandatory: Please choose ACM 2012 classifications from https://dl.acm.org/ccs/ccs_flat.cfm 

%\keywords{Dummy keyword}%TODO mandatory; please add comma-separated list of keywords

%\category{}%optional, e.g. invited paper

\relatedversion{}%optional, e.g. full version hosted on arXiv, HAL, or other respository/website
%\relatedversion{A full version of the paper is available at \url{...}.}

\supplement{}%optional, e.g. related research data, source code, ... hosted on a repository like zenodo, figshare, GitHub, ...

%\funding{(Optional) general funding statement \dots}%optional, to capture a funding statement, which applies to all authors. Please enter author specific funding statements as fifth argument of the \author macro.


\nolinenumbers %uncomment to disable line numbering

\hideLIPIcs  %uncomment to remove references to LIPIcs series (logo, DOI, ...), e.g. when preparing a pre-final version to be uploaded to arXiv or another public repository

%Editor-only macros:: begin (do not touch as author)%%%%%%%%%%%%%%%%%%%%%%%%%%%%%%%%%%
\EventEditors{John Q. Open and Joan R. Access}
\EventNoEds{2}
\EventLongTitle{42nd Conference on Very Important Topics (CVIT 2016)}
\EventShortTitle{CVIT 2016}
\EventAcronym{CVIT}
\EventYear{2016}
\EventDate{December 24--27, 2016}
\EventLocation{Little Whinging, United Kingdom}
\EventLogo{}
\SeriesVolume{42}
\ArticleNo{23}
%%%%%%%%%%%%%%%%%%%%%%%%%%%%%%%%%%%%%%%%%%%%%%%%%%%%%%

\usepackage{gensymb}
\newcommand*{\listsofb}[1]{\mathcal{L} (#1)}
\newcommand*{\listsof}{\mathcal{L}~}

\newcommand{\None}{\emptyset}
\newcommand{\Some}[1]{\degree\kern-0.5ex#1}

\newcommand*{\match}{\textbf{match}~}
\newcommand*{\withl}{\textbf{[}~}
\newcommand*{\withr}{~\textbf{]}}
\newcommand*{\withm}{\quad\textbf{|}\quad}
\newcommand*{\llet}{\textbf{let}~}
\newcommand*{\lin}{\textbf{in}~}
\newcommand{\ITE}[3]{\textbf{if}~{#1}~\textbf{then}~{#2}~\textbf{else}~{#3}}

\newcommand{\Type}{\textsf{\bfseries T}}
\newcommand{\Prop}{\textsf{\bfseries P}}
\newcommand{\bool}{\textsf{B}}
\newcommand{\btrue}{\mathsf{T}}
\newcommand{\bfalse}{\mathsf{F}}
\newcommand{\andb}{\&\&}
\newcommand{\orb}{||}
\newcommand{\notb}{!}
\newcommand{\nat}{\mathsf{N}}
\newcommand{\natS}{1 + }
\newcommand{\length}[1]{|#1|}

\newcommand{\con}{\mathop{{+}\!\!\!{+}}}
\newcommand{\rev}{\mathsf{rev}}
\newcommand{\opt}[1]{\mathcal{O}{#1}}
\newcommand{\nil}{[]}

\newcommand{\eqb}[2]{#1\overset{?}{=}#2}

\newcommand{\bnfmid}{~\mid~}

\newcommand{\concat}{\con}


\newcommand{\TODO}[1]{}

\begin{document}

\maketitle

%TODO mandatory: add short abstract of the document
\begin{abstract}
\end{abstract}

We present and justify a number of considerations made when developing abstractions on our road to a formalisation of Cook's Theorem in Coq. 
Over the last few years, a number elegant reductions in the area of computability theory have been verified in Coq. On the other hand, there haven't been any formalisations of even basic results from computational complexity theory until now. 
It is our goal to change this -- yet, our first formalisation of a polynomial-time reduction from \textbf{3SAT} to \textbf{Clique} has demonstrated the need for suitable layers of abstraction in order to make proofs comprehensible and technically pleasing.

\paragraph*{What makes the computability theory results elegant (and why can't we imitate this)?}
The key to this question lies in the fact that Coq enables synthetic computablity theory: the theory of Coq is constructed such that every definable function is computable. Positive results, such as reductions between problems, can thus be proved directly in Coq, without the need for an abstract machine model.

Coq has many mechanisms easing the construction of reductions: Using $\Sigma$-types and the proof mode, functions can be constructed implicitly together with suitable correctness properties. One thus never needs to explicitly spell out the reduction. Automation can be used to further simplify the construction. 

For complexity-theoretic results, we cannot walk a similar path: we need an abstract machine model (in our case the call-by-value $\lambda$-calculus L) on which our reduction needs to be executed. While there is automation to ease this construction, such as a certifying extraction mechanism from Coq to L that also provides time bounds, we always have to explcitly write down all relevant functions. It is not possible to specify correctness properties together with the function definition, and the proof mode cannot be used to make the construction implicit. This severely limits the elegance achievable.

\paragraph*{How can we nevertheless obtain elegant constructions?}
It seems to be clear that there is no way around explicit constructions and separate verification. Nevertheless, we strive for a pleasing formalisation and concise proofs. 

Many reductions admit a nice ``gadget structure'', i.e.\ parts of the contructed problem instance which, on their own, serve a specific purpose (for instance, one can identify cell gadgets and tape gadgets in the proof of Cook's theorem). 
The idea now is to exploit this structure, by offering a declarative and compositional way of defining gadgets. 

We believe that a nice formalisation can be achieved using record types, containing a function computing the gadget, information on how to interface with the gadget (in order to allow for composition), size and time bounds as well as correctness relations.
Still, the details aren't all that clear at this point. A major challenge is that all but the most primitive gadgets directly depend on the reduction input -- thus, they cannot be computed statically, but their computation procedure needs to be extractable to L. 

\paragraph*{A benchmark}
We therefore propose the following problem, designed by F. Kunze, as a first benchmark in developing suitable tools:
\[\textsf{BinTest}_F (m, a, b) := (F~a~b = \btrue) \land a, b < m,  \] 
where $F : \nat \rightarrow \nat \rightarrow \bool$ is a binary test on numbers. This can be extended to other types, e.g.\ finite sequences of numbers, where we the size bound given by $m$ then has to be understood as a bound on the encoding size of the lists. 

The goal now is to reduce the problem $\textsf{BinOp}_F$, parameterised over $F$, to \textbf{SAT}. While \textsf{BinOp} is simplistic in nature, it still captures some of the main difficulties of proving Cook's theorem: we have some abstract Boolean test $F$ which takes a similar role as the transition function of a Turing machine, and need to express this function using a CNF, without having static access to a function-lookup-table. Moreover, we have to find a succinct way of encoding finite types (in this case a subset of $\nat$) using gadgets in a CNF, composing such gadgets, and express relations between those gadgets. 

Finally, one may even see this problem as part of the construction in the proof of Cook's theorem if one takes $F$ as a relation between two successive tape states, by generalising from natural numbers to sequences of elements of finite types.




\bibliography{memo}{}


\end{document}
