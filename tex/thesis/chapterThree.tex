\newcommand{\varBound}{\textsf{varBound}}
\newcommand{\maxVar}{\textsf{maxVar}}

\chapter{Reducing $k$-SAT to Clique}\label{chap:ksat_clique}
In this chapter, we give a first formalisation of a polynomial-time reduction from the $k$-\SAT{} problem to the \Clique{} problem on undirected graphs. The main purpose of this chapter is to introduce the style of proofs we use. In particular, we comment on some of the more technical aspects of doing the running time analysis in Coq so that we can focus on the less technical aspects in the later chapters. 

The satisfiablity problem \SAT{} on CNFs is well-known: given a Boolean formula in conjunctive normal form, we determine whether there exists a satisfying assignment to the variables. $k$-\SAT{} restricts the definition to CNFs with a clause size of $k$ for a fixed $k$.
\Clique{} is a graph-based problem. Given an undirected graph $G = (V, E)$, a $k$-clique is a set of vertices $C \subseteq V$ such that $\length{C} = k$ and all possible edges between vertices of $C$ are present in $E$. For a graph $G$ and a number $k$, the \Clique{} problem asks whether there exists a $k$-clique in the graph $G$. 

The reduction from $k$-\SAT{} to \Clique{} is an example of a relatively simple reduction: in~\cite{Bläser:TISkript}, it is one of the first examples of a polynomial-time reduction in an undergraduate course on theoretical computer science. There, the full proof on paper does not even span half a page. 
In particular because of this apparent simplicity on paper, we deem this reduction to be an interesting first exploration of the techniques we will later use on a larger scale.

\section{Satisfiability of Conjunctive Normal Forms (CNF)}\label{sec:sat}
\setCoqFilename{Undecidability.L.Complexity.Problems.SAT}
We start by formalising the satisfiablity problem \SAT{}.
A literal is either a Boolean variable or its negation. The disjunction of a number of literals is a clause. Clauses can be combined conjunctively to form a conjunctive normal form. 

\newcommand*{\bvar}{\textsf{var}}
\newcommand*{\literal}{\textsf{literal}}
\newcommand*{\clause}{\textsf{clause}}
\newcommand*{\cnf}{\textsf{cnf}}
\newcommand*{\assgn}{\textsf{assgn}}
\newcommand*{\eval}{\mathcal{E}}
\newcommand*{\evalA}[2]{\eval~#1~#2}
\begin{align*}
  v : \bvar &\defeq \nat \mnote[var]{\bvar}\\
  l : \literal &\defeq \bool \times \nat \mnote[literal]{\literal}\\
  C : \clause &\defeq \listsof{\literal} \mnote[clause]{\clause}\\
  N : \cnf &\defeq \listsof{\clause} \mnote[cnf]{\cnf}
\end{align*}
Here, the Boolean $b$ of a literal $(b, v)$ denotes the literal's \textit{sign}. If the sign is negative (i.e.\ $b = \bfalse$), the literal represents $v$'s negation, otherwise it represents $v$.

An assignment $a : \mnote[assgn]{\assgn}$ to a CNF assigns a Boolean value to each of its variables. We choose to model an assignment as the list of variables which are assigned the value $\btrue$. All other variables are implicitly assigned the value $\bfalse$: $\assgn \defeq \listsof{\bvar}$.

\setCoqFilename{Undecidability.L.Complexity.Problems.SharedSAT}
One can define evaluation functions \mnotec{$\eval$} for \coqlink[evalVar]{variables},
\setCoqFilename{Undecidability.L.Complexity.Problems.SAT}
\coqlink[evalLiteral]{literals}, \coqlink[evalClause]{clauses}, and \coqlink[evalCnf]{CNFs} in a straightforward way. 
We will mainly use the function $\eval : \assgn \rightarrow \cnf \rightarrow \bool$ for CNFs, but refer to the other functions using the same identifier. 
Which function we mean will always be clear from the context and the used metavariables. 

We have the following characterisations of evaluation:
\begin{lemma}[Evaluation Equivalences]\label{lem:cnf_eval_equiv}\leavevmode
  \setCoqFilename{Undecidability.L.Complexity.Problems.SharedSAT}
  \begin{enumerate}
    \coqitem[evalVar_in_iff] $\evalA{a}{v} = \btrue \leftrightarrow v \in a$ \setCoqFilename{Undecidability.L.Complexity.Problems.SAT}
    \coqitem[evalLiteral_var_iff] $\evalA{a}{(b, v)} = \btrue \leftrightarrow \evalA{a}{v} = b$
    \coqitem[evalClause_literal_iff] $\evalA{a}{C} = \btrue \leftrightarrow \exists l \in C, \evalA{a}{l} = \btrue$
    \coqitem[evalCnf_clause_iff] $\evalA{a}{N} = \btrue \leftrightarrow \forall C \in N, \evalA{a}{C} = \btrue$
  \end{enumerate}
\end{lemma}

We say that $a$ satisfies the CNF $N$, denoted \mnotec[satisfies]{$a \models N$}, if $\evalA{a}{N} = \btrue$. 
A similar notation is used for the satisfaction of clauses, literals and variables.

Now, we are able to define the problem \SAT{}:\mnote[SAT]{\SAT{}}
\begin{definition}[Satisfiability of CNFs][SAT]
  $\SAT{}~N \defeq \exists a, a \models N$
\end{definition}

In the sequel, we show that \SAT{} is in \NP{}, which we prove by giving a verifier for \SAT{}. A sensible choice for a certificate of a CNF $N$ being in \SAT{} is a satisfying assignment. Thus a verifier is easy to define: given a CNF $N$ and an assignment $a$, it checks whether $a$ satisfies $N$.
In order to prove the verifier correct, we must show that there exists an assignment of a size which is polynomial in $N$ for every satisfiable formula $N$.
As a list $a$ containing the variables to which $\btrue$ is assigned may contain duplicates and variables which are not even used by the CNF, not every satisfying assignment has a polynomial size.
Therefore, we introduce the notion of \emph{small} assignments. 

\newcommand{\varsOfLiteral}{\textsf{varsOfLiteral}}
\newcommand{\varsOfClause}{\textsf{varsOfClause}}
\newcommand{\varsOfCnf}{\textsf{varsOfCnf}}
\begin{definition}[Used Variables]
  We can define a function $\varsOfCnf : \cnf \rightarrow \listsof{\bvar}$ \mnote[varsOfCnf]{\varsOfCnf} calculating a list of variables contained in a CNF.\
  %We can define functions
  %\begin{itemize}
    %\item $\varsOfLiteral : \literal \rightarrow \listsof{\bvar}$,\mnote[varsOfLiteral]{\varsOfLiteral}
    %\item $\varsOfClause : \clause \rightarrow \listsof{\bvar}$, \mnote[varsOfClause]{\varsOfClause}
    %\item $\varsOfCnf : \cnf \rightarrow \listsof{\bvar}$ \mnote[varsOfCnf]{\varsOfCnf}
  %\end{itemize}
  %calculating a list of variables contained in a literal, clause, or CNF.\
\end{definition}

\newcommand{\smallA}{\textsf{small}}
\newcommand{\dupfree}{\textsf{dupfree}}
\begin{definition}[Small Assignments][assignment_small]
  \mnote[assignment_small]{$\smallA~N~a$}
  An assignment $a$ is small with respect to a CNF $N$ if it is duplicate-free and only contains variables used by the CNF:\ 
  \[ \smallA~N~a \defeq \dupfree~a \land a \subseteq \varsOfCnf~N,\]
  where $\dupfree : \forall X, \listsof{X} \rightarrow \Prop$ is a predicate capturing the absence of duplicates.
\end{definition}

\newcommand{\compressA}{\textsf{compress}}
\newcommand{\dedup}{\textsf{dedup}}
\newcommand{\intersect}{\textsf{intersect}}

Given an arbitrary assignment $a$ for a CNF $N$, we can compute another assignment $a'$ which is small with respect to $N$ by removing duplicates and variables not occuring in the CNF.\
To that end, we assume functions $\coqlink[dedup]{\dedup} : \forall (X : \eqType), \listsof{X} \rightarrow \listsof{X}$ and $\coqlink[intersect]{\intersect} : \forall (X : \eqType), \listsof{X} \rightarrow \listsof{X} \rightarrow \listsof{X}$ where \dedup{} removes duplicates from a list and \intersect{} yields a list that does exactly contain those elements that occur in both of its input lists.
  %\dedup &: \assgn \rightarrow \assgn\mnote{\dedup} \\
  %\dedup~\nil &\defeq \nil \\
  %\dedup~(x::A) &\defeq \ITE{x \overset{?}{\in} A}{\dedup~A}{x::\dedup~A} \\[0.7em]
  %\intersect &: \assgn \rightarrow \assgn \rightarrow \assgn \mnote{\intersect}\\
  %\intersect~\nil~B &\defeq \nil \\
  %\intersect~(x::A)~B &\defeq \ITE{x \overset{?}{\in} B}{x::\intersect~A~B}{\intersect~A~B} \\[0.7em]
\begin{align*}
  \compressA &: \cnf \rightarrow \assgn \rightarrow \assgn\mnote[compress]{\compressA}\\
  \compressA~N~a &\defeq \dedup~(\intersect~a~(\varsOfCnf~N))
\end{align*}
Semantically, \compressA{} changes the assignment: after removing assignments to unused variables, $\bfalse$ is implicitly assigned to these unused variables. However, as the CNF does not refer to these variables, this does not matter. The compression of an assignment is always with respect to a particular CNF.

%\begin{proposition}[Correctness of \dedup]\leavevmode
  %\begin{enumerate}
    %\item $x \in \dedup~a \leftrightarrow x \in a$
    %\item $\dupfree~(\dedup~a)$
  %\end{enumerate}
%\end{proposition}

%\begin{proposition}[Correctness of \intersect]
  %\[x \in \intersect~a~b \leftrightarrow x \in a \land x \in b \]
%\end{proposition}

\begin{lemma}[Correctness of \compressA]\leavevmode
  \begin{enumerate}
    \coqitem[compressAssignment_small] $\smallA~N~(\compressA~N~a)$
    \coqitem[compressAssignment_var_eq] $v \in \varsOfCnf~N \rightarrow \evalA{a}{v} = \evalA{(\compressA~N~a)}{v}$
    \coqitem[compressAssignment_cnf_equiv] $a \models N \leftrightarrow \compressA~N~a \models N$
  \end{enumerate}
\end{lemma}

\begin{lemma}[\SAT{} is in \NP{}][sat_NP]\label{lem:sat_in_np}
  $\SAT{} \in \normalfont{\NP{}}$
\end{lemma}

In order to prove Lemma~\ref{lem:sat_in_np}, we need to show that there are valid certificates of polynomial size exactly for those CNFs $N$ for which $\SAT{}~N$ holds. 
Moreover, it has to be shown that the verifier runs in polynomial time. 
As these proofs are quite technical, we usually omit them on paper. However, as the analyses are an integral part of this work, the following section is an exception and treats the general procedure we use.

\subsection{Running-time Analysis Using the Extraction Mechanism}
  We explain the methods we use to prove polynomial time and size bounds. 
  As an example, we look at the evaluation function $\eval$ for CNFs. 
  Formally, the function is defined as follows:
  \begin{align*}
    \eval_N &: \assgn \rightarrow \cnf \rightarrow \bool \\
    \eval_N~a~\nil &\defeq \btrue \\
    \eval_N~a~(C::N) &\defeq \eval_C~a~C~\&\&~\eval_N~a~N
  \end{align*}
  Here, we have annotated the evaluation function with a subscript to make clearer which one we mean: $\eval_N$ is the evaluation function for CNFs, while $\eval_C$ is the function for clauses.

  In order to extract this function to L, we first need to generate the encodings for the used datatypes, in this case for lists, $\nat$, and $\bool$\footnote{This has to be done only once and then never again.}. 
  Next, all the used functions need to be extracted first, in this case the evaluation of clauses and the Boolean conjunction. We assume this has already been done. 

  Now, the evaluation function itself can be extracted. During the process, the following recurrences for the running time are generated:
  \begin{align}
    T(\eval_N)~a~\nil &\ge 9 \\
    T(\eval_N)~a~(C :: N) &\ge T(\eval_N)~a~N + T (\eval_C)~a~C + 22
  \end{align}
  $T(\cdot)$ can be read as ``time-of''. The numerical constants appearing in the inequalities are the numbers of reduction steps of the extracted function.
  We do not solve the recurrences directly, but instead define a recursive function $T(\eval_N)$ implementing the equations.

  Next, we aim to bound the function $T(\eval_N)$ by a monotonic polynomial $p_{\eval_N} : \nat \rightarrow \nat$ in the encoding size of the arguments such that 
  \[\forall a~N, T(\eval_N)~a~N \le p_{\eval_N}(\encsize{\overline{a}} + \encsize{\overline{N}}) \]
  holds. In principle, we could also use a multivariate polynomial having one variable for each of the arguments, but this would only complicate things. 
  In the case of $\eval_N$, the polynomial $\coqlink[poly__evalCnf]{p_{\eval_N}}(n) \defeq n \cdot p_{\eval_C}(n) + c_{\eval_N} \cdot (n +1)$ works, where $p_{\eval_C}$ is the polynomial bounding $T(\eval_C)$ and $c_{\eval_N}$ is a constant. The bounding equation is then proved by induction on the CNF $N$. 

  Higher-order functions can be treated in a similar way, but there the running time bounds are conditional on bounds for the argument functions. In many cases, however, it seems to be easier to define a function via direct recursion instead of using a higher-order functions which is more complicated to bound.

  Recall that, regarding the size, we only need to bound the size of the result. The procedure of deriving a polynomial is similar in that case, although the extraction mechanism does not derive space bounds currently; instead, we derive bounds via semantic arguments.

  \begin{remark}[Binary versus Unary Numbers]
    In complexity theory, the question of the specific encoding of a problem is an important one. Especially for numbers, this is a delicate subject: a unary encoding of numbers is exponentially larger than the binary encoding. Therefore, one usually uses a binary encoding of numbers. Some number-theoretic problems, for instance, are not \NP{}-hard anymore if a unary encoding instead of a binary encoding is used.
    In this thesis, however, we exclusively use a unary encoding, mainly because it is much easier to work with unary numbers in Coq, although a binary encoding would be possible. 
    For all of the problems we study, using unary numbers is actually sound. Intuitively, this is the case since the numbers do not play a central part in the definition: in \SAT{}, numbers are only used to represent variables. If $n$ variables are used by a CNF, only the numbers up to $n$ need to be used by the encoding.
  \end{remark}

\newcommand{\kCNF}[1]{\text{$#1$-\textsf{CNF}}}
\subsection{k-\SAT{}}
\setCoqFilename{Undecidability.L.Complexity.Problems.kSAT}
Finally, as a variant of \SAT{}, we consider the problem $k$-\SAT{} where each clause consists of exactly $k$ literals. 

We inductively define a predicate stating this property.
\begin{align*}
  \infer{\kCNF{k}~\nil}{} \qquad \infer{\kCNF{k}~(C :: N)}{\length{C} = k \quad \kCNF{k}~N}\mnote[kCNF]{\kCNF{k}}
\end{align*}

The $k$-\SAT{} problem additionally requires that $k > 0$. This is an arbitrary choice as the problem is trivially polynomial-time decidable if no clause contains a literal.\footnote{In this case, $N$ is a yes-instance iff $N = \nil$.}
\begin{definition}[$k$-\SAT{}][kSAT]
  $\text{$k$-\SAT{}}~N \defeq \kCNF{k}~N \land k > 0 \land \SAT{}~N \mnote[kSAT]{$k$-\SAT{}}$
\end{definition}

\begin{remark}
Note that $k$ acts as a parameter to the problem and is not part of an instance. More precisely, we might say that $k$-\SAT{} is a \emph{class} of problems, where we get the problem $k$-\SAT{} for every $k : \nat$. 
\end{remark}

As $\kCNF{k}~N$ can easily be decided in polynomial-time, we obtain a reduction from $k$-\SAT{} to \SAT{}. For the case that $k = 0$ or that $N$ is not a \kCNF{k}, we map to a trivial no-instance such as $x_0 \land \lnot x_0$. 
\setCoqFilename{Undecidability.L.Complexity.Reductions.kSAT_to_SAT}
\begin{lemma}[$k$-\SAT{} reduces to \SAT{}][kSAT_to_SAT]
  $\text{$k$-\SAT{}} \redP{} \SAT{}$
\end{lemma}
\begin{proof}
  As explained in Section~\ref{sec:np_basics}, we need to verify that 
  \begin{itemize}
    \item the reduction is correct, i.e.\ satisfies the reduction equivalence, 
    \item runs in polynomial time, 
    \item and produces instances of a size which is polynomial in the input size.
  \end{itemize}
  The first condition is easy to verify, while the other two conditions use the techniques of the previous section.
\end{proof}

\newcommand{\UGraph}{\textsf{UGraph}}
\newcommand{\Edec}{\ensuremath{E_\textsf{dec}}}
\section{Cliques in Undirected Graphs}\label{sec:clique}
The second problem involved in our first reduction is the \Clique{} problem on undirected graphs. We start by presenting our formalisation of undirected graphs.

\setCoqFilename{Undecidability.L.Complexity.Problems.UGraph}
\begin{definition}[Undirected Graphs][UGraph]
  An undirected graph $G = (V, E) : \UGraph$\mnote[UGraph]{\UGraph} consists of a type of vertices $V : \finType$ and an edge relation $E : V \rightarrow V \rightarrow \Prop$ together with the following proofs:
  \begin{itemize}
    \item $E$ is symmetric, i.e. $\forall v_1~v_2 : V, E~v_1~v_2 \leftrightarrow E~v_2~v_1$, 
    \item and $E$ is decidable, that is, there exists a function $\Edec : \forall v_1~v_2 : V, \textsf{dec}~(\{v_1, v_2\} \in E)$, where $\textsf{dec}~p \defeq p + (\lnot p)$ and we write $\{v_1, v_2\} \in E$ for $E~v_1~v_2$.
  \end{itemize}
  We use the notation $\{v_1, v_2\} \in E$ instead of $(v_1, v_2) \in E$ to explicitly express that $E$ is symmetric.
\end{definition}

\begin{remark}\label{rem:ugraph}
  From the perspective of proving properties about graphs, the previous definition provides an extremely convenient formalisation as all elements of type $V$ are automatically a vertex of the graph and the edge relation is just a decidable proposition instead of, say, a list of pairs. 
  However, the definition has a major problem: the propositional part of the definition, i.e.\ the edge relation, is not extractable to L and the extraction of the finite type is at least difficult. Moreover, even if the propositional parts would not cause any trouble, the definition makes no statements about the \emph{time} that is needed to decide the edge relation. 

  We will deal with this issue in Section~\ref{sec:flat_clique} by introducing a ``flat'' first-order encoding of graphs which can be extracted. 
  As it will be much harder to work with that definition, though, we define an equivalence between both definitions so that we can prove the correctness statements using the full propositional version while doing the running-time analysis on the flat version.
\end{remark}

We define the metavariables $G : \UGraph$, $v : V_G$ and $e : E_G$, where the subscript signifies that we mean the elements $V$ and $E$ of the graph $G$. If the graph $G$ is clear from the context, we may omit the subscript. 

\newcommand{\clique}{\textsf{clique}}
\newcommand{\kclique}[1]{\text{$#1$-\clique}} 
\newcommand{\kcliqueind}[1]{\text{$#1$-\textsf{clique}'}}
\setCoqFilename{Undecidability.L.Complexity.Problems.Clique}
Based on this definition of graphs, we can formalise cliques. 
\begin{definition}[Cliques and $k$-Cliques]
  Fix a graph $G : \UGraph$. A list of vertices $L$ is a clique of $G$ if all the edges between its elements are present in the graph.
  \begin{align*}
    \clique{} &: \listsof{V_G} \rightarrow \Prop\mnote[isClique]{\clique{}} \\
    \clique{}~L &\defeq \dupfree~L \land \forall v_1~v_2 \in L, \{v_1, v_2\} \in E_G 
  \end{align*}
  A $k$-clique is a clique of exactly $k$ elements: 
  $\kclique{k}~L \defeq \length{L} = k \land \clique{}~L\mnote[isKClique]{\kclique{k}}$.
\end{definition}

Alternatively, we can define $k$-cliques inductively, obtaining a definition which is more suitable for inductive proofs. 
\begin{align*}
  \infer{\kcliqueind{0}~\nil}{} \qquad \infer{\kcliqueind{(\natS{k})}~(v :: L)}{v \notin L \quad (\forall v' \in L, \{v, v'\} \in E_G) \quad \kcliqueind{k}~L}\mnote[indKClique]{\kcliqueind{k}}
\end{align*}

The \Clique{} problem asks whether there exists a $k$-clique in a graph $G$: 
\begin{definition}[\Clique{}][Clique]
  $\Clique{}~(G, k) \defeq \exists L : \listsof{V_G}, \kclique{k}~L\mnote{\Clique{}}$
\end{definition}
Note that the number $k$ is part of the instance, in contrast to the parameter $k$ of the class $k$-\SAT{}.\footnote{One can show that for every fixed $k$, the \Clique{} problem is decidable in polynomial time.}

\newcommand{\FlatClique}{\textbf{FlatClique}}
\newcommand{\FlatUGraph}{\textsf{FlatUGraph}}
\subsection{First-order Encoding of Graphs}\label{sec:flat_clique}
\setCoqFilename{Undecidability.L.Complexity.Problems.FlatUGraph}
We now present a variant of graphs and the Clique problem which is extractable to L. This representation eliminates the components which cause trouble (see Remark~\ref{rem:ugraph}) in the above definition: the finite type and the propositional edge relation. 
The finite type we replace by a natural number $v$ giving the number of vertices. Valid vertices are natural numbers which are $< v$. The edge relation is replaced by a list of pairs of vertices: $\FlatUGraph \defeq \nat \times \listsof{\nat \times \nat}$.\mnote[FlatUGraph]{\FlatUGraph}

This type does allow instances which do not make any sense syntactically. Therefore, we enforce syntactic constraints externally:
\[\textsf{FlatUGraph\_wf}~(v, e) \defeq (\forall (v_1, v_2) \in e, (v_2, v_1) \in e) \land (\forall (v_1, v_2) \in e, v_1 < v \land v_2 < v)\mnote[fgraph_wf]{\textsf{FlatUGraph\_wf}} \]
One can easily build a Boolean decider for these constraints. 

Of course, we want to connect graphs of type \UGraph{} and flat graphs of type \FlatUGraph{} somehow. We first define some general relations for the representation of finite types by natural numbers. 

\setCoqFilename{Undecidability.L.Complexity.FlatFinTypes}
\begin{definition}[Representation of Finite Types]
  Let $T : \finType$, $t : T$, and $k, n : \nat$. 
  \begin{gather*}
    n \approx T \defeq n = \length{T} \mnote[finRepr]{$n \approx T$}\\
    k \approx' t \defeq k = \findex~t \mnote[finReprEl']{$k \approx' t$}\\
    k \approx_{T, n} t \defeq k = \findex~t \land n \approx T \mnote[finReprEl]{$k \approx_{T, n} t$}
  \end{gather*}
\end{definition}

Using this definition, it is easy to encode type constructors like $\opt{\cdot}$, $\cdot \times \cdot$, and $\cdot + \cdot$, as well as value constructors like $\OSome{\cdot}$, $( \cdot, \cdot)$, and $\textsf{inl}~\cdot$. More generally, we obtain the same closure properties for the flat representation of finite types as for finite types.

\begin{example}[Encoding of Option Types]
  We define $\coqlink[flatOption]{\textsf{flatOpt}}~t \defeq \natS{t}$, $\coqlink[flatSome]{\textsf{flatSome}}~k \defeq \natS{k}$ and $\coqlink[flatNone]{\textsf{flatNone}} \defeq 0$. 
  Assuming that $n \approx T$ and $k \approx_{T, n} t$, we have:
  \begin{align*}
    \coqlink[finReprOption]{\textsf{flatOpt}~n \approx \opt{T}}
    \qquad \coqlink[finReprElNone]{\textsf{flatNone} \approx_{\opt{T}, \textsf{flatOpt}~n} \ONone}
    \qquad \coqlink[finReprElSome]{\textsf{flatSome}~k \approx_{\opt{T}, \textsf{flatOpt}~n}~\OSome{t}} 
  \end{align*}
\end{example}

\setCoqFilename{Undecidability.L.Complexity.Problems.FlatUGraph}
Thus, we can define a representation relation for graphs as follows: 
\begin{definition}[Representation of Graphs][isFlatGraphOf]\label{def:graph_repr}
  Let a graph $G = (V, E) : \UGraph$ and a flat graph $g = (v, e) : \FlatUGraph$ be given. 
  $g$ represents $G$, written \mnotec[isFlatGraphOf]{$g \approx G$}, if: 
  \begin{itemize}
    \item $v \approx V$, 
    \item $\forall (v_1, v_2) \in e, v_1 < v \land v_2 < v \land \exists V_1~V_2, \{V_1, V_2\} \in E \land v_1 \approx' V_1 \land v_2 \approx' V2$
    \item $\forall V_1~V_2 : V, \{V_1, V_2\} \in E \rightarrow (\findex~V_1, \findex~V_2) \in e$
  \end{itemize}
\end{definition}

\setCoqFilename{Undecidability.L.Complexity.Problems.FlatClique}
Cliques can be defined \coqlink[isfKClique]{in the expected way} for flat graphs. 
The only difference to the definition on non-flat graphs is that we have to explicitly require the graph to be syntactically wellformed (see \textsf{FlatUGraph\_wf}). 
The corresponding problem is called \mnotec[FlatClique]{\FlatClique{}}. A proof that \FlatClique{} is contained in \NP{} is \coqlink[FlatClique_in_NP]{straightforward}.

We can then derive an agreement statement for the two definitions: if $(v, e) \approx (V, E)$ and if $l : \listsof{\nat}$ and $L : \listsof{V}$, then $l$ is a clique in $(v, e)$ \coqlink[clique_flat_agree]{if, and only if,} $L$ is a clique in $(V, E)$. Moreover, from a clique for one representation we can compute a clique for the \coqlink[clique_flatten]{other} \coqlink[clique_unflatten]{representation}.
This agreement allows us to reason using the propositional version, but analyse the running time on the flat version. 

\section{Reduction from $k$-\SAT{} to \Clique{}}
Now that we have defined the involved problems, we present the reduction of $k$-\SAT{} to \Clique{}. 
Assume that the CNF is given as 
\[(l_{0, 0} \lor \ldots \lor l_{0, k-1}) \land \ldots \land (l_{n-1, 0} \lor \ldots \lor l_{m-1, k-1}), \]
for a number of clauses $m$.
The basic idea of the reduction is to create a graph with vertices $(i, j)$ for $i \in [0..m-1]$ and $j \in [0..k-1]$ such that we have a vertex $(i, j)$ for every literal $l_{i, j}$. The edges between the vertices corresponding to two literals encode whether the literals can be satisfied at the same time and belong to different clauses. As a CNF is satisfied if, and only if, there exists a satisfied literal for every clause, the satisfiability problem then reduces to checking whether there exists a $m$-clique in the constructed graph. 
In the following, we present this construction and its proof of correctness formally.

We start with the reduction to the non-flat version \Clique{} using graphs of type \UGraph{}. In order to prove polynomial-time computability, we then transfer the results to the flat version \FlatClique{}. 

For the rest of this section, let us fix a number $k$ with $k > 0$ and a $k$-CNF $N$.\footnote{We will later handle $k = 0$ and instances not satisfying $\kCNF{k}~N$ as special cases.}

\setCoqFilename{Undecidability.L.Complexity.Reductions.kSAT_to_Clique}
In order to define the graph, we need to be able to index into a CNF to obtain the clauses and literals at particular positions. 
We use the operations $N[i] : \opt{\clause}$\mnote[cnfGetClause]{$N[i]$} to obtain the clause at index $i$ and $C[j] : \opt{\literal}$ to obtain the $j$-th literal of a clause $C$.\mnote[clauseGetLiteral]{$C[j]$}
From this, we derive $N[i, j]$ as a way to get the $j$-th literal of the $i$-th clause of the CNF $N$. The operations return option values in order to handle the case that the indices are invalid.\mnote[cnfGetLiteral]{$N[i, j]$}

\newcommand{\literalsConflict}{\textsf{conflict}}
Now, we can set up the graph $G : \UGraph$ in the following way: 
\begin{align*} 
  V_G &\defeq F_{\length{N}} \times F_{k}\mnote[Vcnf]{$V_G$} \\
  \literalsConflict~(s_1, v_1)~(s_2, v_2) &\defeq v_1 = v_2 \land s_1 \neq s_2\mnote[literalsConflict]{\literalsConflict}\\
  E_G ((i_1, j_1), (i_2, j_2)) &\defeq i_1 \neq i_2 \land \mnote[Ecnf]{$E_G$}
  \begin{aligned}[t]
    \forall l_1~l_2, &(N[\findex{i_1}, \findex{j_1}] = \OSome{l_1} \\
                     &\land N[\findex{i_2}, \findex{j_2}] = \OSome{l_2}) \\
                     &\rightarrow~\lnot \literalsConflict~l_1~l_2 
  \end{aligned}
\end{align*}
Here, the type $F_n$ is a finite type consisting of $n$ elements. Two literals $l_1, l_2$ are conflicting if they refer to the same variable but have different signs. 
The definition of the edge relation then captures exactly the intuition described above: two vertices are connected if their corresponding literals are not conflicting and belong to different clauses. 
Note that the way in which we handle the case that $N[\findex{i}, \findex{j}] = \ONone$ does not matter: by construction, there will always exist a literal at these positions since we assumed that $N$ is in $k$-CNF.

\begin{remark}
  Labelling the graph with the syntactic literals occuring in the CNF instead of the positions would not work: a single literal can occur multiple times in different clauses (or even in the same clause) while the construction requires us to differentiate these different occurences. 
\end{remark}

One can easily check that $E_G$ is \coqlink[Ecnf_symm]{symmetric} and \coqlink[Ecnf_dec]{decidable}.
For the correctness of the reduction, our goal is to show that $\SAT{}~N \leftrightarrow \Clique{}~(G, \length{N})$. 

\subsection{From Assignments to Cliques}
First of all, let us assume that $a_\textsf{sat}$ is a satisfying assignment for $N$, i.e.\ $a_\textsf{sat} \models N$. 
We have to derive the existence of a $\length{N}$-clique in the graph $G$. The idea is to pick an arbitrary literal which is satisfied by $a_\textsf{sat}$ for every clause. The corresponding vertices in $G$ form a clique by the construction of the edge relation. 

To prove this formally, we require a few facts. 
\begin{fact}[][evalLiteral_true_no_conflict]\label{fact:literal_conflict}
  Let $a : \assgn$ and $l_1, l_2 : \literal$ be arbitrary. 
  \[\evalA{a}{l_1} = \btrue \rightarrow \evalA{a}{l_2} = \btrue \rightarrow \lnot \literalsConflict{}~l_1~l_2 \]
\end{fact}

\begin{fact}[][literalInClause_exists_vertex]\label{fact:litInClause_exists_vertex}
  Let $C$ be the $i$-th clause of $N$, i.e. $N[i] = \OSome{C}$, and $l \in C$. Then:
  \[\exists (ci,li) : V_G, i = \findex{ci} \land C[\findex{li}] = \OSome{l}\]
  That is, there exists a vertex of the graph that corresponds to a literal which is syntactically equal to $l$ and is part of the same clause as $l$.
\end{fact}
Note that we could strengthen the preceding fact by also fixing the position of the literal inside the clause. However, for the proof it is only relevant that the literal belongs to a specific clause.

We prove the existence of a clique by an induction over the number of clauses. In order for an induction to go through, the statement needs to be strengthened: as we have fixed a graph for the whole CNF, we also require that the clique provided by the inductive hypothesis only uses vertices belonging to the smaller CNF. Moreover, we need that the corresponding literals are satisfied by $a_{\textsf{sat}}$ in order to derive the existence of the necessary edges if we add a vertex for the new clause in the inductive step. Thus, we arrive at the following statement:

\begin{lemma}[Existence of a Clique][exists_clique']
  Assume a sub-CNF $N'$ of $N$ such that $N = N'' \concat N'$ for some $N''$. Then there exists $L : \listsof{V_G}$ with: 
  \begin{enumerate}
    \item $\kclique{\length{N'}}~L$, 
    \item $\forall (ci, li) \in L, \findex{ci} \ge \length{N''}$, 
    \item and $\forall (ci, li) \in L, \exists l, N[\findex{ci}, \findex{li}] = \OSome{l} \land \evalA{a_\textsf{sat}}{l} = \btrue$.
  \end{enumerate}
\end{lemma}
\begin{proof}
  By induction on $N'$ using Facts~\ref{fact:litInClause_exists_vertex} and~\ref{fact:literal_conflict} in the cons case.
  %By induction on $N'$. In the base case, we choose $L = \nil$ and are done. 
  %In the successor case, we have $N = N'' \concat (C :: N'$ for a clause $C$. By the inductive hypothesis with $N''$ instantiated with $N'' \concat [C]$, we get a $\length{N'}$-clique $L$ such that the corresponding literals are satisfied by $a_\textsf{sat}$ (3.) and the clause indices are at least $\natS{\length{N''}}$ (2.). 
  %By Lemma~\ref{lem:cnf_eval_equiv} and $a_{\textsf{sat}} \models (C :: N)$, there exists $l \in C$ with $\evalA{a_{\textsf{sat}}}{l} = \btrue$. 
  %We apply Fact~\ref{fact:litInClause_exists_vertex} for $i \defeq \length{N''}$ to $l$ and $C$ to obtain a new vertex $v$ for the clause $C$. 
  %The list $v :: L$ is the clique we seek. By the choice of $v$, it is clear that the literal corresponding to $v$ is satisfied and thus 3.\ holds. Similarly 2.\ is preserved by this choice. 
  %It remains to show that $v :: L$ is indeed a clique. For that pupose, we switch to the inductive characterisation $\kcliqueind{(\natS{\length{N'}})}~(v :: L)$, which can be proved using the inductive hypothesis and Fact~\ref{fact:literal_conflict}.
\end{proof}

\begin{corollary}[][exists_clique]
  There exists a $\length{N}$-clique $L$ in $G$.
\end{corollary}

\subsection{From Cliques to Assignments}
Conversely, we have to show that, given a $\length{N}$-clique $L$ in $G$, there exists a satisfying assignment for $N$. The proof proceeds in four steps:
\begin{enumerate}
  \item For every clause of $N$, there exists one corresponding vertex in $L$.
  \item Therefore, there exists a list of (clause, literal)-indices $(i, j)$ where every clause is mentioned exactly once and the referenced literals are non-conflicting. 
  \item If we map this list of positions to a list of syntactic literals, these literals are non-conflicting and as there is at least one literal per clause, it suffices to satisfy these literals in order to satisfy the whole CNF.
  \item Finally, the list of literals can be turned into a satisfying assignment.
\end{enumerate}

Only the proof of the first step is interesting, the other steps are straightforward.
We use an instance of the pigeonhole principle: 
\setCoqFilename{Undecidability.L.Complexity.Reductions.Pigeonhole}
\begin{lemma}[16.8.7 in~\cite{iclnotes}][pigeonhole']\label{lem:pigeonhole}
  Let $A \subseteq B$ with $\length{B} < \length{A}$. Then $\lnot \dupfree~A$. 
\end{lemma}

\setCoqFilename{Undecidability.L.Complexity.Reductions.kSAT_to_Clique}
\begin{lemma}[][clique_vertex_for_all_clauses]
  For every clause of the CNF, there exists a corresponding vertex in $L$:
  \[\forall i < \length{N}, \exists (ci, li), (ci, li) \in L \land \findex{ci} = i \]
\end{lemma}
\begin{proof}
  As for every $i$ and $L$ the goal is decidable, we can constructively do a proof by contradiction. Thus, assume that $\forall (ci, li) \in L, \findex{ci} \neq i$. It suffices to show that the list of clause indices of $L$ contains duplicates, i.e. $\lnot (\dupfree \withl ci \withm (ci, li) \in L \withr)$, as there is only an edge between two vertices $(ci_1, li_1)$ and $(ci_2, li_2)$ if $ci_1 \neq ci_2$ by definition of $E_G$. 
  We apply Lemma~\ref{lem:pigeonhole} for $B \defeq [f_0, \ldots, f_{\length{N} -1}] \setminus f_i$, where $f_k$ is the $k$-th element of the finite type $F_{\length{N}}$, as $i$ is not contained in the list of clause indices by assumption. 
\end{proof}

%\newcommand{\satPositions}{\textsf{satPositions}}
%Continuing with the second step, we define the list of (clause, literal)-positions as 
%\[\satPositions \defeq \withl(\findex{ci}, \findex{li}) \withm (ci, li) \in L \withr. \]

%Formalising the above intuitions, we get the following results:
%\begin{lemma}[Properties of \satPositions]\leavevmode
  %\begin{enumerate}
    %\item $\dupfree~\satPositions$
    %\item $\forall i < \length{N}, \exists j, (i, j) \in \satPositions$
    %\item $(i, j) \in \satPositions \rightarrow \exists l, N[i, j] = \OSome{l}$
    %\item Assume $(i_1, j_1), (i_2, j_2) \in \satPositions$ with $(i_1, j_1) \neq (i_2, j_2)$. Then there are $l_1, l_2$ with $N[i_1, j_1] = \OSome{l_1} \land N[i_2, j_2] = \OSome{l_2} \land \lnot (\literalsConflict~l_1~l_2)$. 
  %\end{enumerate}
%\end{lemma}

%\newcommand{\satLiterals}{\textsf{satLiterals}}
%By part 3 of the previous lemma, we can safely map the positions to syntactic literals.
%\[\satLiterals \defeq \withl l \withm (i, j) \in \satPositions \land N[i, j] = \OSome{l} \withr \]
%The list of literals contains at least one literal per clause and is non-conflicting. 

%\begin{lemma}[Properties of \satLiterals]\label{lem:satLiterals}\leavevmode
  %\begin{enumerate}
    %\item $\forall C \in N, \exists l \in C, l \in \satLiterals$
    %\item Assume an assignment $a$ with $\forall l \in \satLiterals, \evalA{a}{l} = \btrue$. Then $a \models N$. 
    %\item $\forall l_1~l_2 \in \satLiterals, \lnot (\literalsConflict~l_1~l_2)$
  %\end{enumerate}
%\end{lemma}

%Finally, we can construct an assignment from the list of literals by mapping the positive literals to $\btrue$ and the negative literals to $\bfalse$. Let $a_\textsf{sat}$ be such an assignment satisfying $v \in a_{\textsf{sat}} \leftrightarrow (\btrue, v) \in \satLiterals$. It follows that $\forall l \in \satLiterals, \evalA{a_\textsf{sat}}{l} = \btrue$. By part 2 of Lemma~\ref{lem:satLiterals}, $a_\textsf{sat}$ satisfies $N$.
We omit the formal proof of the other steps in this presentation. In the end, we \coqlink[exists_assignment]{obtain a satisfying assignment} $a_{\textsf{sat}}$ for $N$. 

Combining both directions, we get a reduction $f : \cnf \rightarrow \UGraph \times \nat$ if we map CNFs which are not in $k$-CNF to a trivial no-instance.
\begin{theorem}[Correctness of the Reduction][reduction_to_Clique]
  $\text{$k$-\SAT{}}~N \leftrightarrow \Clique{}~(f~N)$
\end{theorem}

However, this reduction cannot be extracted to $L$ as we are using the unextractable type of graphs \UGraph. 
Therefore, we derive a reduction to the flat version \FlatClique{}. The construction of the graph mirrors the definition of $E_G$, where we use the flat constructors for pairs to construct the vertices, see Section~\ref{sec:flat_clique}. 
This graph is connected to the non-flat graph $G$ by the relation $\approx$ (Definition~\ref{def:graph_repr}). 
We obtain the following polynomial-time reduction:
\setCoqFilename{Undecidability.L.Complexity.Reductions.kSAT_to_FlatClique}
\begin{theorem}[][kSAT_to_FlatClique_poly]
  $\textsf{$k$-\SAT{}} \redP{} \FlatClique{}$
\end{theorem}

\begin{remark}
  A first version of this reduction~\cite{memo_clique} did not use a separate non-extractable version of graphs but instead directly proved correctness on the flat version. 
  The mechanisation was much more tedious because, among other things, we always had to work with the possibility that a vertex is invalid, i.e.\ not actually part of the graph. 
  The new proof halved the lines of code of the Coq mechanisation and is also much more understandable. Therefore, we believe that defining a separate extractable problem is the right way if the natural formalisation of a problem in Coq is not directly extractable.
\end{remark}
